% kapitel1.tex
\chapter{Einleitung}
\label{chapter:kap1}

\section{Motivation}
    Routing-Algorithmen sind essenziell für die effiziente Datenübertragung in Computernetzwerken und helfen dabei, den optimalen Pfad für den Datenfluss zwischen zwei oder mehr Knotenpunkten zu bestimmen. Zu den bekanntesten Algorithmen gehören Dijkstra's Algorithmus, der auf den kürzesten Weg fokussiert, und der Bellman-Ford-Algorithmus, der bei dynamischen Netzwerken mit wechselnden Kosten funktioniert. \\
    In dieser Arbeit haben wir uns das Paper \cite{parham2021traffic} und die zwei dazugehörigen Repositories \cite{original_p1} und \cite{original_p2} angeschaut. Dabei haben wir versucht, die Ergebnisse zu reproduzieren. Auch haben wir uns drei weitere Algorithmen überlegt und diese in das bestehende Projekt eingearbeitet. Unsere Erweiterungen kann man hier finden \cite{reproduktion_p1} und \cite{reproduktion_p2}.


\section{Aufbau der Arbeit}
    Innerhalb dieses Reportes stellen wir unsere Ergebnisse bei der Einarbeitung unserer eigenen drei Algorithmen vor. Dabei ist jedem Algorithmus ein Kapitel gewidmet (\ref{chapter:algorithmus1}, \ref{chapter:algorithmus2}, \ref{chapter:algorithmus3}). Wir erläutern dabei die Idee des Algorithmus, die jeweiligen Vor- und Nachteile, die Schwierigkeiten, welche wir bei der Einarbeitung in die beiden Projekte hatten, und die Ergebnisse, die wir mit den Algorithmen erzielt haben. In Kapitel \ref{chapter:kap5} gehen wir noch auf die Reproduktion der Gruppe 4 ein, welche auch zwei eigene Algorithmen entwickelt hat.  